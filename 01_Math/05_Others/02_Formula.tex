\begin{enumerate}

\item $\sum_{i=1}^n i=\sum_{i=1}^n\varphi(i)\lfloor n/i\rfloor$

\item $a^n\mod(a^k*y)=a^k(a^{n-k}\mod y)$

\item 小于$n$且互素的数之和为$n\varphi(n)/2$

\item 错排公式:$D(n)=(n-1)(D(n-2)+D(n-1))=\sum_{i=2}^n\frac{(-1)^kn!}{k!}=[\frac{n!}{e}+0.5]$

\item 欧拉定理推广:$\gcd(n,p)=1\Rightarrow a^n\equiv a^{n\%\varphi(p)}\pmod p$

\item 模的幂公式:$a ^ n \pmod {m} = 
\begin{cases}
a ^ n \mod m & n < \varphi(m)\\
a ^ {n \% \varphi(m) + \varphi(m)} \mod m & n \ge \varphi(m)
\end{cases}
$

\item 皮克定理:$S=a+b/2-1$ S:面积,a:内部格点数,b:边上格点数

\item 约瑟夫环:$F[1]=0,F[i]=(F[i-1]+m)\% i$

\item 位数公式:正整数$x$的位数$N=\log_{10}(n)+1$

\item 斯特灵公式$n!\approx\sqrt{2\pi n}(\frac{n}{e})^n$

\item 设$a>1,m,n>0$,则$\gcd(a^m-1,a^n-1)=a^{\gcd(m,n)}-1$

\item 设$a>b,\gcd(a,b)=1$,则$\gcd(a^m-b^m,a^n-b^n)=a^{\gcd(m,n)}-b^{\gcd(m,n)}$

$$
G=\gcd(C_n^1,C_n^2,...,C_n^{n-1})=
\begin{cases}
	n, & \text{$n$ is prime} \\
	1, & \text{$n$ has multy prime factors} \\
	p, & \text{$n$ has single prime factor $p$}
\end{cases}
$$

$\gcd(Fib(m),Fib(n))=Fib(\gcd(m,n))$

\item 若$\gcd(m,n)=1$,则:

\begin{enumerate}
\item 最大不能组合的数为$m*n-m-n$
\item 不能组合数个数$N=\frac{(m-1)(n-1)}{2}$
\end{enumerate}

\item $(n+1)lcm(C_n^0,C_n^1,...,C_n^{n-1},C_n^{n})=lcm(1,2,...,n+1)$

\item 若$p$为素数,则$(x+y+...+w)^p\equiv x^p+y^p+...+w^p\pmod p$

\item 卡特兰数:1, 1, 2, 5, 14, 42, 132, 429, 1430, 4862, 16796, 58786, 208012

$h(0)=h(1)=1,h(n)=\frac{(4n-2)h(n-1)}{n+1}=\frac{C_{2n}^n}{n+1}=C_{2n}^n-C_{2n}^{n-1}$

\item 伯努利数:$B_n = -\frac{1}{n+1} \sum_{i=0}^{n-1} C_{n+1}^i B_i$

$$\sum_{i=1}^n i^k = \frac{1}{k+1} \sum_{i=1}^{k+1}C_{k+1}^i B_{k+1-i}(n+1)^i$$

\item 二项式反演:$$f_n = \sum_{i = 0} ^ n (-1) ^ i \binom{n}{i} g_i \Leftrightarrow g_n = \sum_{i = 0} ^ n (-1) ^ i \binom{n}{i} f_i$$
$$f_n = \sum_{i = 0} ^ n \binom{n}{i} g_i \Leftrightarrow g_n = \sum_{i = 0} ^ n (-1) ^ {n - i} \binom{n}{i} f_i$$

\item 多重求和:对于第K重求和 $A[i]=A[i-1]*(K+i-1)/i$

\item 拉格朗日四平方和定理:每个正整数均可表示为4个整数的平方和

\item 图论欧拉公式:$V-E+F=1+k$,$k$为连通分量

\item 球缺公式:$V=\frac{\pi h^2(3r-h)}{3}$

\item 矩阵树定理:度数矩阵减邻接矩阵的的任意一个代数余子式,有向图;外向图:度数改入度;内向图:度数改出度;有向图去掉的行列必须是根节点对应的那个。

\item 当 $x\geq\phi(p)$ 时有 $a^x\equiv a^{x \; mod \; \phi(p) + \phi(p)}\pmod p$

\item $\mu^2(n)=\sum_{d^2|n} \mu(d)​$

\item $\sum_{d|n} \varphi(d)=n$

\item $\sum_{d|n} 2^{\omega(d)}=\sigma_0(n^2)$,其中 $\omega$ 是不同素因子个数

\item $\sum_{d|n} \mu^2(d)=2^{\omega(d)}$

\item 杜教筛

求 $S(n)=\sum_{i=1}^n f(i)$,其中 $f$ 是一个积性函数。

构造一个积性函数 $g$,那么由 $(f*g)(n)=\sum_{d|n}f(d)g(\frac{n}{d})$,得到 $f(n)=(f*g)(n)-\sum_{d|n,d<n}f(d)g(\frac{n}{d})$。

\begin{eqnarray}
g(1)S(n)&=&\sum_{i=1}^n (f*g)(i)-\sum_{i= 1}^{n}\sum_{d|i,d<i}f(d)g(\frac{n}{d}) \\
&\overset{t=\frac{i}{d}}{=}& \sum_{i=1}^n (f*g)(i)-\sum_{t=2}^{n} g(t) S(\lfloor \frac{n}{t} \rfloor)
\end{eqnarray}


当然,要能够由此计算 $S(n)$,会对 $f,g$ 提出一些要求:

\begin{enumerate}
	
	\item $f*g$ 要能够快速求前缀和。
	
	\item $g$  要能够快速求分段和(前缀和)。
	
	\item 对于正常的积性函数 $g(1)=1$,所以不会有什么问题。

\end{enumerate}

在预处理 $S(n)$ 前 $n^{\frac{2}{3}}$ 项的情况下复杂度是 $O(n^{\frac{2}{3}})$。

\item 数论函数求和

\begin{enumerate}

	\item $\sum_{i=1}^n i[gcd(i, n)=1] = \frac {n \varphi(n) + [n=1]}{2}$
	
	\item $\sum_{i=1}^n \sum_{j=1}^m [gcd(i,j)=x]=\sum_d \mu(d) \lfloor \frac n {dx} \rfloor  \lfloor \frac m {dx} \rfloor$
	
	\item $\sum_{i=1}^n \sum_{j=1}^m gcd(i, j) = \sum_{i=1}^n \sum_{j=1}^m \sum_{d|gcd(i,j)} \varphi(d) = \sum_{d} \varphi(d) \lfloor \frac nd \rfloor \lfloor \frac md \rfloor$
	
	\item $S(n)=\sum_{i=1}^n \mu(i)=1-\sum_{i=1}^n \sum_{d|i,d < i}\mu(d) \overset{t=\frac id}{=} 1-\sum_{t=2}^nS(\lfloor \frac nt \rfloor)$,利用 $[n=1] = \sum_{d|n} \mu(d)$
	
	\item $S(n)=\sum_{i=1}^n \varphi(i)=\sum_{i=1}^n i-\sum_{i=1}^n \sum_{d|i,d<i} \varphi(i)\overset{t=\frac id}{=} \frac {i(i+1)}{2} - \sum_{t=2}^n S(\frac n t)$,利用 $n = \sum_{d|n} \varphi(d)$
	
	\item $\sum_{i=1}^n \mu^2(i) = \sum_{i=1}^n \sum_{d^2|n} \mu(d)=\sum_{d=1}^{\lfloor \sqrt n \rfloor}\mu(d) \lfloor \frac n {d^2} \rfloor$ 
	
	\item $\sum_{i=1}^n \sum_{j=1}^n gcd^2(i, j)= \sum_{d} d^2 \sum_{t} \mu(t) \lfloor \frac n{dt} \rfloor ^2 \\
	  \overset{x=dt}{=} \sum_{x} \lfloor \frac nx \rfloor ^ 2 \sum_{d|x} d^2 \mu(\frac tx)$
	
	\item $\sum_{i=1}^n \varphi(i)=\frac 12 \sum_{i=1}^n \sum_{j=1}^n [i \perp j] - 1=\frac 12 \sum_{i=1}^n \mu(i) \cdot\lfloor \frac n i \rfloor ^2-1$

\end{enumerate}

\item Fibonacci

\begin{enumerate}
	
	\item $F_{a+b}=F_{a-1} \cdot F_b+F_a \cdot F_{b+1}$
	
	\item $F_1+F_3+\dots +F_{2n-1} = F_{2n},F_2 + F_4 + \dots + F_{2n} = F_{2n + 1} - 1$
	
	\item $\sum_{i=1}^n F_i = F_{n+2} - 1$
	
	\item $\sum_{i=1}^n F_i^2 = F_n \cdot F_{n+1}$
	
	\item $F_n^2=(-1)^{n-1} + F_{n-1} \cdot F_{n+1}$
	
	\item $gcd(F_a, F_b)=F_{gcd(a, b)}$

\end{enumerate}

\item 生成函数

\begin{enumerate}

	\item $(1+ax)^n=\sum_{k=0}^n \binom {n}{k} a^kx^k$
	
	\item $\dfrac{1-x^{r+1}}{1-x}=\sum_{k=0}^nx^k$
	
	\item $\dfrac1{1-ax}=\sum_{k=0}^{\infty}a^kx^k$
	
	\item $\dfrac 1{(1-x)^2}=\sum_{k=0}^{\infty}(k+1)x^k$
	
	\item $\dfrac1{(1-x)^n}=\sum_{k=0}^{\infty} \binom{n+k-1}{k}x^k$
	
	\item $e^x=\sum_{k=0}^{\infty}\dfrac{x^k}{k!}$
	
	\item $\ln(1+x)=\sum_{k=0}^{\infty}\dfrac{(-1)^{k+1}}{k}x^k$

\end{enumerate}

\item 莫比乌斯反演

\begin{enumerate}
	
	\item $g(n) = \sum_{d|n} f(d) \Leftrightarrow f(n) = \sum_{d|n} \mu (d) g( \frac{n}{d})$
	
	\item $f(n)=\sum_{n|d}g(d) \Leftrightarrow g(n)=\sum_{n|d} \mu(\frac{d}{n}) f(d)$

\end{enumerate}

\item 低阶等幂求和

\begin{enumerate}
	
	\item $\sum_{i=1}^{n} i^{1} = \frac{n(n+1)}{2} = \frac{1}{2}n^2 +\frac{1}{2} n​$
	
	\item $\sum_{i=1}^{n} i^{2} = \frac{n(n+1)(2n+1)}{6} = \frac{1}{3}n^3 + \frac{1}{2}n^2 + \frac{1}{6}n$
	
	\item $\sum_{i=1}^{n} i^{3} = \left[\frac{n(n+1)}{2}\right]^{2} = \frac{1}{4}n^4 + \frac{1}{2}n^3 + \frac{1}{4}n^2$
	
	\item $\sum_{i=1}^{n} i^{4} = \frac{n(n+1)(2n+1)(3n^2+3n-1)}{30} = \frac{1}{5}n^5 + \frac{1}{2}n^4 + \frac{1}{3}n^3 - \frac{1}{30}n​$
	
	\item $\sum_{i=1}^{n} i^{5} = \frac{n^{2}(n+1)^{2}(2n^2+2n-1)}{12} = \frac{1}{6}n^6 + \frac{1}{2}n^5 + \frac{5}{12}n^4 - \frac{1}{12}n^2$
	
\end{enumerate}

\end{enumerate}